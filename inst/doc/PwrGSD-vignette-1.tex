\documentclass{article}
\usepackage{harvard}
\usepackage{hyperref}
\usepackage{graphics}
\usepackage{amsmath}
\usepackage{indentfirst}
\usepackage[utf8]{inputenc}
\usepackage[top=1in, bottom=1in, left=1in, right=1in]{geometry}
\DeclareMathOperator{\var}{var}
% \VignetteIndexEntry{Using PwrGSD}

\usepackage{/usr/local/lib64/R/share/texmf/Sweave}
\begin{document}

\title{Using PwrGSD to compute Operating Characteristics of a candidate monitoring scheme at
  a specified hypothetical trial scenario (Version 1.15)}
\author{Grant Izmirlian}
\maketitle

\section{Introduction}
The function $\mathbf{PwrGSD}$ computes several operating characteristics, such as
power, expected duration and relative risks at the stopping boundaries, given
a specification of the interim monitoring scheme and choice of test statistic
under a specified hypothetical progression of the trial. The capabilities of 
$\mathbf{PwrGSD}$ allow for 
\begin{itemize}
  \item[1]{{\bf Non-proportional hazards alternatives}
      via the specification of trial arm specific piecewise constant hazard rates, 
      piecewise exponential survival functions, or the stipulation of one of these
      in arm 0 together with a piecewise constant hazard ratio for the main endpoint.}
  \item[2]{{\bf Flexible specification of the censoring distribution}
      The trial arm specific censoring distributions may be specified via piecewise constant
      hazard or piecewise exponential survival functions.}
  \item[3]{{\bf Two modes of non-compliance per each of the two trial arms}
      Each form of non-compliance is stipulated via a waiting time distribution, specified via
      piecewise constant hazards or piecewise exponential survival, together with a post-noncompliance
      main endpoint distribution, also specified via hazards or survival functions.}      
  \item[4]{{\bf Choice of test statistic}
      Currently, the asymptotic method of calculation supports several members of the weighted log-rank
      family of statistics: Fleming-Harrington wieghts of given exponents $FH(g, \rho)$, a
      variant $SFH(g,\rho,x)$ which is equal to the $FH(g, \rho)$ function but stopped at the value
      attained at $x$, or the $Ramp(x)$ function, which has linear rise from zero to its maximum
      value, attained at $x$ and then constant weight thereafter.  The simulation method of calculation
      supports all of these plus the integrated survival difference statistic.}    
  \item[5]{{\bf Choice of boundary construction method:}
      Currently either Lan-Demets with a variety of possible spending functions
      \begin{itemize}
        \item[i]{O'Brien-Fleming}
        \item[ii]{Pocock}
        \item[iii]{Wang-Tsiatis Power Family}
      \end{itemize}
      The Haybittle method is also supported in the case of efficacy bounds only}
  \item[6]{{\bf Efficacy bounds only or simultaneous calculation of efficacy and futility bounds}}
\end{itemize}

The goal of this vignette is to understand the features and capabilities of {\bf PwrGSD} by
trying several examples. 

In the first example, we compute power at a specific alternative, \verb`rhaz`, under an interim
analysis plan with roughly one analysis per year, some crossover between intervention and
control arms, with Efficacy and futility boundaries constructed via the Lan-Demets
procedure with O'Brien-Fleming spending. We investigate the behavior of three weighted log-rank
statistics: (i) the Fleming-Harrington(0,1) statistic, (ii) a stopped version of the F-H(0,1)
statistic capped off at 10 years, and (iii) the deterministic weighting function with linear
increase between time 0 and time 10 with constant weight thereafter.

\begin{Schunk}
\begin{Sinput}
> tlook <- c(7.14, 8.14, 9.14, 10.14, 10.64, 11.15, 12.14, 
+     13.14, 14.14, 15.14, 16.14, 17.14, 18.14, 19.14, 
+     20.14)
> t0 <- 0:19
> h0 <- c(rep(0.000373, 2), rep(0.000745, 3), rep(0.00149, 
+     15))
> rhaz <- c(1, 0.9125, 0.8688, 0.7814, 0.6941, 0.6943, 
+     0.6072, 0.5202, 0.4332, 0.652, 0.6524, 0.6527, 0.653, 
+     0.6534, 0.6537, 0.6541, 0.6544, 0.6547, 0.6551, 0.6554)
> hc <- c(rep(0.0105, 2), rep(0.0209, 3), rep(0.0419, 15))
> hd1B <- c(0.1109, 0.1381, 0.1485, 0.1637, 0.2446, 0.2497, 
+     0)
\end{Sinput}
\end{Schunk}

\begin{Schunk}
\begin{Sinput}
> library(PwrGSD)
> example.1 <- PwrGSD(EfficacyBoundary = LanDemets(alpha = 0.05, 
+     spending = ObrienFleming), FutilityBoundary = LanDemets(alpha = 0.1, 
+     spending = ObrienFleming), RR.Futility = 0.82, sided = "1<", 
+     method = "A", accru = 7.73, accrat = 9818.65, tlook = tlook, 
+     tcut0 = t0, h0 = h0, tcut1 = t0, rhaz = rhaz, tcutc0 = t0, 
+     hc0 = hc, tcutc1 = t0, hc1 = hc, tcutd0B = c(0, 13), 
+     hd0B = c(0.04777, 0), tcutd1B = 0:6, hd1B = hd1B, 
+     noncompliance = crossover, gradual = TRUE, WtFun = c("FH", 
+         "SFH", "Ramp"), ppar = c(0, 1, 0, 1, 10, 10))
\end{Sinput}
\end{Schunk}

In the next example, we construct the efficacy boundary using the stochastic curtailment procedure.
\begin{Schunk}
\begin{Sinput}
> example.2 <- example.1
> example.2$call$EfficacyBoundary <- as.call(expression(SC, 
+     alpha = 0.05, crit = 0.9))
> example.2 <- update(example.2)
\end{Sinput}
\end{Schunk}

\end{document}
