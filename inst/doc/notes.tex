\documentclass{article}
\usepackage{nopageno}
\usepackage{amsxtra}
\usepackage{harvard}
\usepackage{hyperref}
%begin: author's definitions---pls-do-not-move-or-modify------%
\font\headf=cmbx12 scaled\magstep1                            %
\def\P{{{\rm I}\kern-.18em{\rm P}}}                           %
\def\E{{{\rm I}\kern-.18em{\rm E}}}                           %
\def\e{\mathrm e}                                             %
\def\hateq{\;\;\widehat =\;\;}                                %
\accentedsymbol{\betahathat}{\widehat{\widehat \beta}}        %
\newcommand{\eqind}{\buildrel {\rm\bf D} \over = }            %
%end: author's definitions----------thank-you-----------------%

\topmargin -0.75in
\textheight 8.5in
\textwidth 6in
\oddsidemargin 0in
\hoffset=0.275in

\setcounter{page}{1}

\begin{document}

\title{{Notes on the PwrGSD Package}}

\author{{Grant Izmirlian}\footnotemark[1]~\footnotemark[2]}

\footnotetext[1]{National Cancer Institute; Executive Plaza North, Suite 3131;
6130 Executive Blvd, MSC 7354; Bethesda, MD 20892-7354}
\footnotetext[2]{This article is a U.S. Government work and is in the public domain in the
  U.S.A.}

\maketitle
%\cgsn{Publishing Arts Research Council}{98--1846389}

These are notes detailing the algorithm, analogous to that of Armitage and McPhearson,
that is used to calculate the stopping boundaries and the power using asymptotic 
methods. We assume that there could be two separate information scales, one for spending
the type I and II error probabilities, and the other for internally scaling the observed
values of the statistic and boundaries to a brownian motion with a drift.



